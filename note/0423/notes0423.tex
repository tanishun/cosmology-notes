\documentclass[12pt,a4paper]{article}
\usepackage[height=8.8in,width=6.45in]{geometry}

%\usepackage{amsmath}
\usepackage{mathtools} %extended amsmath
\usepackage{amssymb}


% not understanded
\countdef\cpart=1
\def\part#1{%
\newpage
\advance\cpart by 1
\section*{Part \the\cpart: #1}
\addcontentsline{toc}{section}{Part \the\cpart: #1}
%
}
\usepackage[nottoc]{tocbibind}


\usepackage{bm} %boldmath
\usepackage{mathdots}

\mathtoolsset{showonlyrefs=true,showmanualtags}%represent the number of equation when quote eqref

\usepackage{graphicx} % include graphic
\usepackage{float}
\usepackage[svgnames]{xcolor}%(xcolor.pdf p38~39)

\usepackage{tensor}%for part 1 relativity

\usepackage{enumerate}
\usepackage{adjustbox}
\usepackage[dvipdfmx,
 colorlinks=true,
 bookmarks=true,
 bookmarksnumbered=true,
 pdftitle={notes},% タイトル
 pdfsubject={relativity},% サブタイトル
 pdfauthor={Shun Taniwaki},% 著者
 pdfkeywords={Weinberg-Cosmology}% キーワード
 citecolor=DarkGreen,
 linkcolor=FireBrick,
 %urlcolor=FireBrick,
 linktocpage=true,
]{hyperref}
\usepackage{pxjahyper}

\usepackage[thref,amsmath,thmmarks]{ntheorem}

\theoremstyle{plain}
\theorembodyfont{\normalfont}
%\theoremsymbol{\text{\normalfont (Q.E.D.)}}
\theoremseparator{.}
\theoremstyle{break}

\usepackage{mdframed}
\newmdtheoremenv[ntheorem,backgroundcolor=black!10,linecolor=black!0]{principle}{原理}[section]
\newmdtheoremenv[ntheorem,backgroundcolor=black!10,linecolor=black!0]{assumption}{仮定}[section]
\newmdtheoremenv[ntheorem,backgroundcolor=black!10,linecolor=black!0]{definition}{定義}[section]
\newmdtheoremenv[ntheorem,backgroundcolor=black!10,linecolor=black!0]{formula}{公式}[section]
%\newmdtheoremenv[ntheorem,backgraoudcolor=black!0]{theorem}{定理}[subsection]
\newmdtheoremenv[ntheorem,backgroundcolor=black!10]{theorem}[definition]{事実}
\newmdtheoremenv[ntheorem,backgroundcolor=black!3,linecolor=black!0]{example}{例}[definition]
\newmdtheoremenv[ntheorem,backgroundcolor=black!3,linecolor=black!0]{object}{適用する対象}[definition]
\newmdtheoremenv[ntheorem,backgroundcolor=black!0]{unclear}{Unclear points}[section]
%\newmdtheoremenv[mdframedoption]{envname}[numberedlike]{caption} [within]







% def bold faces
\bmdefine{\bfa}{a}
\bmdefine{\bfb}{b}
\bmdefine{\bfc}{c}
\bmdefine{\bfd}{d}
\bmdefine{\bfe}{e}
\bmdefine{\bff}{f}
\bmdefine{\bfg}{g}
\bmdefine{\bfh}{h}
\bmdefine{\bfi}{i}
\bmdefine{\bfj}{j}
\bmdefine{\bfk}{k}
\bmdefine{\bfl}{l}
\bmdefine{\bfm}{m}
\bmdefine{\bfn}{n}
\bmdefine{\bfo}{o}
\bmdefine{\bfp}{p}
\bmdefine{\bfq}{q}
\bmdefine{\bfr}{r}
\bmdefine{\bfs}{s}
\bmdefine{\bft}{t}
\bmdefine{\bfu}{u}
\bmdefine{\bfv}{v}
\bmdefine{\bfw}{w}
\bmdefine{\bfx}{x}
\bmdefine{\bfy}{y}
\bmdefine{\bfz}{z}
\bmdefine{\bfA}{A}
\bmdefine{\bfB}{B}
\bmdefine{\bfC}{C}
\bmdefine{\bfD}{D}
\bmdefine{\bfE}{E}
\bmdefine{\bfF}{F}
\bmdefine{\bfG}{G}
\bmdefine{\bfH}{H}
\bmdefine{\bfI}{I}
\bmdefine{\bfJ}{J}
\bmdefine{\bfK}{K}
\bmdefine{\bfL}{L}
\bmdefine{\bfM}{M}
\bmdefine{\bfN}{N}
\bmdefine{\bfO}{O}
\bmdefine{\bfP}{P}
\bmdefine{\bfQ}{Q}
\bmdefine{\bfR}{R}
\bmdefine{\bfS}{S}
\bmdefine{\bfT}{T}
\bmdefine{\bfU}{U}
\bmdefine{\bfV}{V}
\bmdefine{\bfW}{W}
\bmdefine{\bfX}{X}
\bmdefine{\bfY}{Y}
\bmdefine{\bfZ}{Z}
\bmdefine{\bftheta}{\theta}
\bmdefine{\bfphi}{\varphi}
\bmdefine{\bfomega}{\omega}

%mathbf
\newcommand{\mba}{\mathbf{a}}
\newcommand{\mbb}{\mathbf{b}}
\newcommand{\mbc}{\mathbf{c}}
\newcommand{\mbd}{\mathbf{d}}
\newcommand{\mbe}{\mathbf{e}}
\newcommand{\mbf}{\mathbf{f}}
\newcommand{\mbg}{\mathbf{g}}
\newcommand{\mbh}{\mathbf{h}}
\newcommand{\mbi}{\mathbf{i}}
\newcommand{\mbj}{\mathbf{j}}
\newcommand{\mbk}{\mathbf{k}}
\newcommand{\mbl}{\mathbf{l}}
\newcommand{\mbm}{\mathbf{m}}
\newcommand{\mbn}{\mathbf{n}}
\newcommand{\mbo}{\mathbf{o}}
\newcommand{\mbp}{\mathbf{p}}
\newcommand{\mbq}{\mathbf{q}}
\newcommand{\mbr}{\mathbf{r}}
\newcommand{\mbs}{\mathbf{s}}
\newcommand{\mbt}{\mathbf{t}}
\newcommand{\mbu}{\mathbf{u}}
\newcommand{\mbv}{\mathbf{v}}
\newcommand{\mbw}{\mathbf{w}}
\newcommand{\mbx}{\mathbf{x}}
\newcommand{\mby}{\mathbf{y}}
\newcommand{\mbz}{\mathbf{z}}
\newcommand{\mbA}{\mathbf{A}}
\newcommand{\mbB}{\mathbf{B}}
\newcommand{\mbC}{\mathbf{C}}
\newcommand{\mbD}{\mathbf{D}}
\newcommand{\mbE}{\mathbf{E}}
\newcommand{\mbF}{\mathbf{F}}
\newcommand{\mbG}{\mathbf{G}}
\newcommand{\mbH}{\mathbf{H}}
\newcommand{\mbI}{\mathbf{I}}
\newcommand{\mbJ}{\mathbf{J}}
\newcommand{\mbK}{\mathbf{K}}
\newcommand{\mbL}{\mathbf{L}}
\newcommand{\mbM}{\mathbf{M}}
\newcommand{\mbN}{\mathbf{N}}
\newcommand{\mbO}{\mathbf{O}}
\newcommand{\mbP}{\mathbf{P}}
\newcommand{\mbQ}{\mathbf{Q}}
\newcommand{\mbR}{\mathbf{R}}
\newcommand{\mbS}{\mathbf{S}}
\newcommand{\mbT}{\mathbf{T}}
\newcommand{\mbU}{\mathbf{U}}
\newcommand{\mbV}{\mathbf{V}}
\newcommand{\mbW}{\mathbf{W}}
\newcommand{\mbX}{\mathbf{X}}
\newcommand{\mbY}{\mathbf{Y}}
\newcommand{\mbZ}{\mathbf{Z}}
%tilde
\newcommand{\tilg}{\tilde{g}}
\newcommand{\tilGamma}{\tilde{\Gamma}}
%****
\def\TODO#1{{\color{FireBrick} #1}}
%***
\newcommand{\Slash}[1]{{\ooalign{\hfil/\hfil\crcr$#1$}}}%スラッシュ引くよう

\def\ii{\mathrm{i}}
\def\Nequals#1{$\mathcal{N}{=}\,#1$}

%def tensors
\newcommand{\tensorR}[1]{\tensor{R}{ #1}}
\newcommand{\tensorGamma}[1]{\tensor{\Gamma}{ #1}}
\newcommand{\tensorG}[1]{\tensor{G}{ #1}}
\newcommand{\tensorg}[1]{\tensor{g}{ #1}}

%def math operators
\DeclareMathOperator{\arcsinh}{arcsinh}
\def\odi#1#2{\frac{d #1}{d #2}} %ordinary differential
\def\todi#1#2{\frac{d^{2} #1}{d #2^{2}}} %two ordinary differential
\newcommand{\pdi}[2]{\frac{\partial #1}{\partial #2}}%partial differential
\newcommand{\tpdi}[2]{\frac{\partial #1^{2}}{\partial #2^{2}}}%two partial differential
\def\vev#1{\Bigl(#1\Bigr)}% Vacume Expectation Value ?
\DeclareMathOperator{\sign}{sign}% sign
\def\diag{\mathop{\mathrm{diag}}\nolimits}% diagonalize
\def\tr{\mathop{\mathrm{tr}}\nolimits}% trace
\def\adj{\mathop{\mathrm{adj}}}% adjoint




%*********************************************************************
\begin{document}
\setcounter{part}{0}
\setcounter{section}{3}
\subsection*{B. 二次の指標距離(Secondary distance indicators)}
これまでの距離指標は、特異測度が膨張速度に比べて無視できるような大きなred shift $(z> 0.03)$の距離を測定するほど、十分に明るくない。
したがって、大きなred shift $(z> 0.03)$に対して必要である。
これは、セファイド(Cephids)変光星より明るいもの、つまり、銀河(galaxies)そのものか、もしくは超新星(supernovae)のような銀河そのものと同じくらい明るいものではなくてはならない。
\begin{object}
  二次の距離指標は、Cephids 変光星より明るいものに対して使う。
\end{object}

長年に渡って、セファイド変光星はたかだか200$\sim$300 万pc (2-3 Mpc)までの距離指標としてしか使えなかった、
今まで、セファイド変光星の適用範囲は、
\begin{itemize}
  \item 局所銀河群\\
        (銀河系とアンドロメダ星雲 M31,およびM33, LMC, SMCのような1ダースかそこらの小さな銀河からなる)
  \item 地球の近傍の銀河群\\
        (M81 銀河団、M101 銀河団、ちょうこくしつ座(Sculptor)銀河団)
\end{itemize}
これでは、適当な銀河や超新星の距離の”較正”に十分でない。
(球状星団、$\mathrm{H_{II}}$領域(水素の電離領域)、銀河で最も明るい星)
様々な”中間的”距離の指標が必要となった。
しかしながら、これは過去のお話。
現在では、ハッブル宇宙望遠鏡のおかげで、約$30\,\rm{Mpc}$の距離に渡って、セファイド変光星を観測できるようになった。
そのため、二次の距離指数を中間的距離指数を用いることなしで、直接検証することができるようになる。

\subsubsection{二次の指標距離(Secondary distance indicators)の例}
以下では、これまで開発されている二次の指標距離の主要な例を4つ:
\begin{enumerate}
  \item Tully \,-\, Fisher 関係
  \item Faber \,-\, Jacson 関係
  \item 基準平面(Fundamental plane)
  \item Ia 型超新星 (Type Ia supernovae)
\end{enumerate}
と、ハッブル定数を測った例:
\begin{enumerate}
\renewcommand{\labelenumi}{\arabic{enumi}.}
\setcounter{enumi}{4}
  \item 表面輝度の揺らぎ(Surface brightness fluctuations)
\end{enumerate}
の計5つを紹介する。

\subsubsection*{1.Tally\,-\,Fisher 関係}
銀河は非常に遠くにあっても見つけることができる。
しかしながら、同じ絶対光度を持つような銀河の種類は今の所見つかっていない。
1977年、タリーとフィッシャーは、「ある適当な種類の渦巻銀河の絶対光度を評価する手法」を編み出した。(タリーフィッシャー関係)
\footnote{なお、これは経験論的(?)}

水素原子の二つの微細構造間の遷移から生じる$(\lambda = ) 21\,\mathrm{cm}$吸収線のスペクトル幅に着目した。\footnote{1s エネルギー準位の分裂によ微細構造の準位である。}
ある適当な種類の渦巻銀河における$21\,\mathrm{cm}$吸収線は、銀河開店から生じるドップラー効果によって、スペクトル幅が広がっている。
吸収線のスペクトル幅$:W$は、
\begin{itemize}
  \item 銀河の軌道速度の情報があったり、
  \item 銀河の質量と相関があったり、
  \item 銀河の絶対光度と相関があったり
\end{itemize}
する。
なお、タリーフィッシャー関係は、電波の周波数帯域に現れる一酸化炭素の分子の遷移のような他のスペクトルの線幅にも適用できるので、$\lambda = 21 \, \mathrm{cm}$に特有ではない。
\begin{flushleft}
\textgt{Tully\,-\,fisher のアプローチの応用}
\end{flushleft}
ある$21 \, \mathrm{cm}$線幅に対し、
赤外線長域の絶対光度(実際には、絶対光度$\times$ある共通の定数)を与えるような関数$L_I(W)$を定義して、較正することを考えよう。
この関数$L_I(W)$の形状が、(時は1997年、)
「21個の銀河団にある555この渦巻銀河のサンプル(なお、大部分が赤方偏移が0.01より小さいもの)」より、$L_I(W) \propto W^3$と見出された。
(なお、誤差については、銀河の持つ特異速度のために、ここの銀河のペアに関して求められた絶対光度の日は無視できない誤差を含んでいる。
しかし、555この銀河サンプルから測定された絶対光度の相対値を、最小二乗法で滑らかな曲線に近似することにより、そのような誤差は相殺される。)
規格化は、セファイド変光星を含む距離のわかっている(たかだか、25\,Mpc\,の距離にある)15個の渦巻銀河)の絶対光度を与えることで求められた。

このように較正された関数$L_I(W)$を、赤方偏移が0.013から0.03の範囲($0.013<z<0.03$)、すなわち近距離、にある14個の銀河団にある銀河の距離を求めるのに使って、ハッブル定数を求めることができた。(これらの赤方偏移は、特異測度を完全に無視できるほど十分に大きくないかもしれないけれども、非常に多くの銀河を用いることでこの問題というのは解消することができる。)このようなやり方で求められたハッブル定数は、
\begin{align}
 H_0 = 70 \pm5 \,\mathrm{km\,s^{-1}\,Mpc^{-1}}
\end{align}•%
であった。

より最近の進展として(2000)、ハッブル宇宙望遠鏡の基幹プロジェクトが、セファイド変光星を用いて、タリーフィッシャー関係を再構成し(ただし、LMCの距離を50 kpcと仮定して)、タリーフィッシャー関係を再構成して、タリーフィッシャー関係から得られた距離を、赤方偏移が0.007から0.03の間にある($0.007<z<0.03$)、19個の銀河団サンプルの赤方偏移と比べて、$H_0$を求めた。
(これらの銀河団のサンプルはジョバネッリらのG97の調査から得られたものである。)
結果、
\begin{align}
 H_0 = 70 \pm3 \pm7 \,\mathrm{km\,s^{-1}\,Mpc^{-1}}
\end{align}•%
と求められた。


\subsubsection*{2.Faber-Jackson 関係}
Tally\,-\,Fisher 関係は、渦巻銀河の軌道速度と絶対光度の相関を与える関係であったのに対して、
Faber \,-\, Jackson 関係は、楕円運動の乱雑速度\footnote{楕円銀河は、回転しているというより星がランダムに動き回っているので、 その星のランダム運動の速度幅が用いられる。}と絶対光度の相関を与える。
この関係は、ビリアルの定理で、直接、乱雑速度の2乗平均と銀河の質量が関係あるという点で、Tally\,-\,Fisher 関係より優れている。

\subsubsection*{3.基準平面(Fundamental plane)}
Faboer-Jackson 法の改良版。
軌道速度と絶対光度の相関が、銀河の表面輝度、すなわち、銀河の面積に依存すると認識されたことによって改良された。
\begin{align}
  H_0 = 78 \pm 5 \text{(統計誤差)} \, \pm 9 \text{(系統誤差)} \, \mathrm{km\,s^{-1}\,Mpc^{-1}}
\end{align}

\subsubsection*{4.Ia 型超新星(Type Ia supernovae)}
Ia 型超新星は、連星系にいる白色矮星がもう片方の星からの物質の降着を受け、その質量が電子の縮退圧で支え得られる最大質量ーチャンドラセカール限界ーに近い値まで増加した際に生じると考えられている。
この時、白色矮星は不安定となり、温度と密度の上昇によって炭素と酸素がニッケル56に転化され、数十億 pc($\sim  O(10^9) \, \mathrm{pc} $)の距離でも見えるような熱核爆発を起こす。
この爆発は、常にチャンドラセカール限界に近い質量でおこるのでその絶対光度にはほとんど変化がなく、ほぼ理想的な距離指標となる。
\footnote{
I型超新星は、そのスペクトルに水素の線が見られないものの呼称である。
}

この関係は、距離のわかっているいくつかの銀河中のIa 型超新星の測定によって構成されている。
1937年から1999年のの間に、セファイド変光星の観測で距離が測定できた銀河中に見つかった超新星は10個であった。
これらのうち、6個のIa 型超新星は、ハッブル宇宙望遠鏡の基幹$H_0$プロジェクトチームによって絶対光度と減光時間の関係を較正するのに用いられた。
そして、この関係は、セロ・トロロ汎米天文台で観測された、赤方偏移が0.01から0.1の銀河における29個のIa型超新星への距離を計算するのに用いられた。
これらの距離を赤方偏移を横軸にした図にすると、ハッブル定数は
\begin{align}
  H_0 = 71 \pm 2 \text{(統計誤差)} \, \pm 6 \text{(系統誤差)} \, \mathrm{km\,s^{-1}\,Mpc^{-1}}
\end{align}
を得ることができる。
この値は、それ以前にハーバード大学のグループがIa 型超新星を用いて決めた値:
\begin{align}
  H_0 = 67 \pm 7 \, \mathrm{km\,s^{-1}\,Mpc^{-1}}
\end{align}
とよく一致している。2005年、このグループのメンバーは、その後新しい結果として、
\begin{align}
  H_0 = 73 \pm 4(\text{(統計誤差)}) \,\pm 5 \text{(系統誤差)} \, \mathrm{km \, s^{-1} \, Mpc^{-1}}
\end{align}

\subsubsection*{5.表面輝度の揺らぎ(fluctuations of surface brightness)}
1998年、Tonry とSchneider は、銀河観測で得られた画像の部分部分における表面輝度の揺らぎを用いて、その銀河までの距離を測定する方法を考案した。
この方法の気持ちは、平均の面輝度は距離によって変わらないが、遠くになるほど単位面積(立体角)中に含まれる星の数が増えるので、平均の周りの揺らぎは小さくなる。
銀河系の周りの球状星団の写真と遠方の楕円銀河の写真を見比べると、球状星団の方が星の粒々が目立って揺らぎが大きいことがわかる。このことを巧みに使うコロンブスの卵のような距離決定方法。

ある銀河中の恒星が、光度に応じたグループ$i$に分けられて、それぞれのグループ$i$に属する全ての星が同じ絶対光度$L_i$を持つとしてみる。
\begin{align}
  l_{I} = \sum_i \frac{N_i L_i}{4\pi d^2}
\end{align}

\begin{align}
  \langle (N_i - \langle N_i \rangle)(N_j - \langle N_j \rangle)\rangle = \delta_{ij} \langle N_i \rangle
\end{align}
$\langle \,\,\,\, \rangle:$銀河の画像中心の小さな部分ごとにわたる平均。

\begin{align}
  \frac{\langle (l-\langle l\rangle)^2 \rangle}{\langle l  \rangle} = \frac{\bar{L}}{4 \pi d^2} \label{eq:1.3.10}
\end{align}
$\bar{L}$は光度で重みをかけて平均をとったの星の平均光度:
\begin{align}
  \bar{L} = \frac{\sum_i \langle N_i \rangle L_i^2}{\sum_i \langle N_i \rangle L_i}
\end{align}
銀河自身の高度に比べれば、異なる銀河に対して変化が小さい。
\begin{align}
  \bar{L} \quad \longrightarrow \text{\eqref{eq:1.3.10}が較正} \longrightarrow \text{距離$d$を求めることができる。}
\end{align}
トンリーらは、セファイド変光星の観測から距離が分かっている銀河の表面輝度の揺らぎを研究することで、$\bar{L}$を等価な$\bar{M_I}$
\begin{align}
  \bar{M_I} = (-1.74 \pm 0.07) + (4.5 \pm 0.25) [m_V -m_I -1.15]
\end{align}
ただし、$m_V - m_I$は銀河の色のパラメータで、紫外線と可視光領域における見かけの等級差に等しい。($1.0 <m_V - m_I<1.5$という仮定がある。)
上の式を用いて、より大きなred shift にある銀河の距離を求め、
\begin{align}
  H_0 = 81 \pm 6 \,\mathrm{km\,s^{-1}\,Mpc^{-1}}
\end{align}

\subsubsection{ハッブル定数を測定する様々な現象}
\begin{itemize}
  \item 他の型の超新星、新星、球状星団、および、惑星状星雲の見かけの光度と絶対光度の比較
  \item 楕円銀河の直径 ー 速度分散関係
  \item 重力レンズ (1.12)
  \item Sunyaev\,-\,Zel'dovich 効果(2.3)
  \item 他いろいろ
\end{itemize}
ハッブル宇宙望遠鏡の基幹$H_0$プロジェクトチームは、
\begin{itemize}
  \item Tally\,-\,Fisher 関係
  \item Ia\,型超新星
  \item そして、他のいくつかの二次指標距離
\end{itemize}
をまとめて
\begin{align}
  H_0 = 71 \pm 6  \,\mathrm{km\,s^{-1}\,Mpc^{-1}}
\end{align}
と結論した。

ハッブル定数を別の観点で考えてみる。7.2節でみるように、宇宙マイクロ波背景放射(CMB)の異方性の研究から、
\begin{align}
  H_0 = 73 \pm 3  \,\mathrm{km\,s^{-1}\,Mpc^{-1}}
\end{align}
が得られている。これは、ここで議論した手法にはよっていないが、いくつかの宇宙論的過程:
\begin{itemize}
  \item 平坦な空間
  \item 時間によらない真空のエネルギー
  \item 冷たい暗黒物質
\end{itemize}
に基づいている。
この理由から、CMB の解析から得られる精度の高い$H_0$の測定は、ここで議論した昔ながらの測定に取って代われることは愛。
むしろ、これらの極めて異なる手法によって得られて$H_0$の値の一致(または、将来的に見つかるかもしれない値の不一致)は、CMB 解析の宇宙論的過程を確認(または棄却)するために有用となろう!

\subsubsection{ハッブル定数の最近の表記について}
ハッブル定数の不定性を考慮するために、最近では大抵、
\begin{align}
  H_0 =  100 h  \,\mathrm{km\,s^{-1}\,Mpc^{-1}}
\end{align}
と書いている。ただし、無次元パラメータ$h \sim 0.7$のあたりとする。
これに対する、ハッブル時間は
\begin{align}
  1/H_0 = 9.778 \times 10^9 h^{-1} \quad \text{年}
\end{align}
である。

\section{Luminosity distances and angular diameter distances}
We must now consider the measurement of distances at large redshifts, say $z > 0.1$,
where the effects of cosmological expansion on the determination of distance can no longer be neglected.
It is these measurements that can tell us whether the expansion of the universe is accelerating or decelerating, and how fast.
Before we can interpret these measurements, we will need to consider in this section how to define distance at large redshifts, and we will have to apply Einstein's field equations to the Robertson-Walker metric in the following section.
After that, we will return in Section 1.6 to the measurements of distances for large redshift, and their interpretation.

In the previous section we derived the familiar relation $l = L/4\pi d^2$ for the apparent luminosity $l$ of a source of absolute luminosity $L$ at a distance $d$. At large distances this derivation needs modification for three reasons:
\begin{enumerate}
  \item
  At the time $t_0$ that the light reaches earth, the proper area of a sphere drawn around the luminous object and passing through the earth is given by the metric (1.1.10)%%eqref
  as $4\pi r_1^2 a^2(t_0)$, where r1 is the coordinate distance of the earth as seen from the luminous object, which is just the same as the coordinate distance of the luminous object as seen from the earth. The fraction of the light received in a telescope of aperture A on earth is therefore $A/4\pi r_1^2 a^2(t_0)$, and so the factor $1/d^2$ in the formula for l must be replaced with $1/r_1^2a^2(t_0)$.
  \item
  The rate of arrival of individual photons is lower than the rate at which they are emitted by the redshift factor $a(t_1)/a(t_0) = 1/(1 + z)$
  \item
  The energy $h\nu_0$ of the individual photons received on earth is less than the energy $h\nu_1$ with which they were emitted by the same redshift factor $1/(1 + z)$.
\end{enumerate}
Putting this together gives the correct formula for apparent luminosity of a source at radial coordinate $r_1$ with a redshift $z$ of any size
\begin{align}
  l=\frac{L}{4 \pi r_{1}^{2} a^{2}\left(t_{0}\right)(1+z)^{2}}
\end{align}
It is convenient to introduce a “luminosity distance” $d_L$, which is defined so that the relation between apparent and absolute luminosity and luminosity distance is the same as Eq. (1.3.3):
\begin{align}
  l=\frac{L}{4 \pi d_{L}^{2}}
\end{align}
Eq. (1.4.1) can then be expressed as:
\begin{align}
  d_{L}=a\left(t_{0}\right) r_{1}(1+z)
\end{align}
For objects with $z\ll1$,we can usefully write the relation between luminosity distance and redshift as a power series.
The redshift $1 + z \equiv a(t_0)/a(t_1)$ is related to the “look-back time” $t_0-t_1$ by:
\begin{align}
  z=H_{0}\left(t_{0}-t_{1}\right)+\frac{1}{2}\left(q_{0}+2\right) H_{0}^{2}\left(t_{0}-t_{1}\right)^{2}+\ldots
\end{align}
where $H_0$ is the Hubble constant (1.2.7) and $q_0$ is the deceleration parameter
\begin{align}
  q_{0} \equiv \frac{-1}{H_{0}^{2} a\left(t_{0}\right)}\left.\frac{d^{2} a(t)}{d t^{2}}\right|_{t=t_{0}}
\end{align}
This can be inverted, to give the look-back time as a power series in the redshift
\begin{align}
  H_{0}\left(t_{0}-t_{1}\right)=z-\frac{1}{2}\left(q_{0}+2\right) z^{2}+\ldots
\end{align}
The coordinate distance $r_1$ of the luminous object is given by Eq. (1.2.2) as:
\begin{align}
  \frac{t_{0}-t_{1}}{a\left(t_{0}\right)}+\frac{H_{0}\left(t_{0}-t_{1}\right)^{2}}{2 a\left(t_{0}\right)}+\cdots=r_{1}+\ldots
\end{align}
with the dots on the right-hand side denoting terms of third and higher order
in $r_1$. Using Eq. (1.4.6), the solution is
\begin{align}
  r_{1} a\left(t_{0}\right) H_{0}=z-\frac{1}{2}\left(1+q_{0}\right) z^{2}+\cdots
\end{align}
This gives the luminosity distance (1.4.3) as a power series
\begin{align}
  d_{L}=H_{0}^{-1}\left[z+\frac{1}{2}\left(1-q_{0}\right) z^{2}+\cdots\right]
\end{align}
We can therefore measure $q_0$ as well as $H_0$ by measuring the luminosity distance as a function of redshift to terms of order $z^2$.
The same reasoning has been used to extend the expression (1.4.9) to fourth order in $z$:
\begin{align}
  d_{L}(z)
  & =H_{0}^{-1}\left[z+\frac{1}{2}\left(1-q_{0}\right) z^{2}-\frac{1}{6}\left(1-q_{0}-3 q_{0}^{2}+j_{0}+\frac{K}{H_{0}^{2} a_{0}^{2}}\right) z^{3}\right. \\
  & +\frac{1}{24}\left(2-2 q_{0}-15 q_{0}^{2}-15 q_{0}^{3}+5 j_{0}+10 q_{0} j_{0}\right.\\
  &+s_{0}+\frac{2 K\left(1+3 q_{0}\right)}{H_{0}^{2} a_{0}^{2}} ) z^{4}+\cdots ]
\end{align}
Years ago cosmology was called “a search for two numbers,” $H_0$ and $q_0$. The determination of $H_0$ is still a major goal of astronomy, as discussed in the previous section. On the other hand, there is less interest now in $q_0$.
Instead of high-precision distance determinations at moderate redshifts, of order 0.1 to 0.2, which would give an accurate value of $q_0$, we now have distance determinations of only moderate precision at high redshifts, of order unity, which depend on the whole form of the function $a(t)$ over the past few billion years.
For redshifts of order unity, it is not very useful to expand in powers of redshift.
In order to interpret these measurements, we will need a dynamical theory of the expansion, to be developed in the next section. As we will see there, modern observations suggest strongly that there are not two but at least three parameters that need to be measured to calculate $a(t)$.

Before turning to this dynamical theory, let’s pause a moment to clarify the distinction between different measures of distance.
So far, we have encountered the proper distance (1.1.15) and the luminosity distance (1.4.3).
There is another sort of distance, which is what we measure when we compare angular sizes with physical dimensions.
Inspection of the metric (1.1.12) shows that a source at co-moving radial coordinate $r_1$ that emits light at time $t_1$ and is observed at present to subtend a small angle $\theta$ will extend over a proper distance s (normal to the line of sight) equal to a(t1)r1θ. The angular diameter distance dA is defined so that $\theta$ is given by the usual relation of Euclidean geometry
\begin{align}
  \theta = s/d_A
\end{align}
and we see that
\begin{align}
  d_A = a(t_1) r_1
\end{align}

Comparison of this result with Eq. (1.4.3) shows that the ratio of the luminosity and angular-diameter distances is simply a function of redshift:
\begin{align}
  d_{A} / d_{L}=(1+z)^{-2}
\end{align}
Therefore if we have measured the luminosity distance at a given redshift (and if we are convinced of the correctness of the Robertson–Walker met- ric), then we learn nothing additional about a(t) if we also measure the angular diameter distance at that redshift.
Neither galaxies nor supernovas have well-defined edges, so angular diameter distances are much less useful in studying the cosmological expansion than are luminosity distances. However, as we shall see, they play an important role in the theoretical analysis of both gravitational lenses in Chapter 9 and of the fluctuations in the cosmic microwave radiation background in Chapters 2 and 7.
We will see in Section 8.1 that the observation of acoustic oscillations in the matter density may allow a measurement of yet another distance, a structure distance, equal to $a(t_0)r_1 = (1 + z)d_A$.




\end{document}
\section{Luminosity distances and angular diameter distances}\label{sec:Luminosity-distances-and-angular-diameter-distances}
\section{Dynamics of expansion}\label{sec:Dynamics-of-expansion}
\section{Distances at large redshift}\label{sec:Distances-at-large-redshift}
\section{Cosmic expansion of tired light?}\label{sec:Cosmic-expansion-of-tired-light?}
\section{Age}\label{sec:Age}
\section{Masses}\label{sec:Masses}
\section{Intergalactic absorption}\label{sec:Intergalactic-absorption}
\section{Number counts}\label{sec:Number-counts}
\section{Quintessence}\label{sec:Quintessence}
\section{Horizons}\label{sec:Horizons}

\part{THE COSMIC MICROWAVE RADIATION BACKGROUND}
\section{Expectations and discovery of the microwave background}
\section{The equilibrium era}
\section{Recombination and last scattering}
\section{The dipole anisotorpy}
\section{The Sunyaev-Zel'dovich effect}
\section{Primary fluctuations in the microwave background: A first look}

\part{THE EARLY UNIVE
